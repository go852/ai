\documentclass{ctexbeamer}
\usetheme{Madrid} % 可选主题
\usepackage{ctex}
\usepackage{amsmath,amssymb}
\usepackage{graphicx}
\usepackage{tcolorbox}

\title{华东师大版七年级上册数学 1.6节}
\subtitle{有理数的加减混合运算} % 根据实际修改
\author{你的名字}
\date{\today}

\begin{document}

% 目录页
\begin{frame}{目录}
    \tableofcontents
\end{frame}

\section{学习目标}
\begin{frame}{学习目标}
    \begin{block}{知识目标}
        \begin{itemize}
            \item 掌握有理数加减混合运算的步骤
            \item 理解去括号法则的符号变化规律
        \end{itemize}
    \end{block}

    \begin{block}{能力目标}
        \begin{itemize}
            \item 能解决含分数的混合运算问题
            \item 培养数学抽象与逻辑推理能力
        \end{itemize}
    \end{block}
\end{frame}

\section{知识点回顾}
\begin{frame}{知识点回顾}
    \begin{exampleblock}{有理数加法法则}
        \begin{align*}
            (+a) + (+b) &= +(a+b) \\
            (-a) + (-b) &= -(a+b) \\
            (+a) + (-b) &= 
            \begin{cases}
                +(a-b) & \text{若 } a > b \\
                -(b-a) & \text{若 } a < b
            \end{cases}
        \end{align*}
    \end{exampleblock}

    \begin{alertblock}{注意}
        减法统一为加法:$a - b = a + (-b)$
    \end{alertblock}
\end{frame}

\section{新知识讲解}
\begin{frame}{混合运算步骤}
    \begin{tcolorbox}[colback=blue!5!white,colframe=blue!75!black]
        \textbf{三步法:}
        \begin{enumerate}
            \item 将减法转换为加法
            \item 去括号(注意符号变化)
            \item 按顺序计算
        \end{enumerate}
    \end{tcolorbox}

    \begin{example}
        \[
        (-3) + 5 - (-2) - 7 = (-3) + 5 + (+2) + (-7)
        \]
        \pause
        \[
        = [(-3) + (-7)] + (5 + 2) = (-10) + 7 = -3
        \]
    \end{example}
\end{frame}

\section{典型例题}
\begin{frame}{例题分析}
    \begin{block}{例题1:温度变化计算}
        某地一周温度变化记录:+3°, -2°, +5°, -4°, +1°, 求总变化量
    \end{block}

    \begin{align*}
        \uncover<2->{&3 + (-2) + 5 + (-4) + 1 \\}
        \uncover<3->{&= (3+5+1) + [(-2)+(-4)] \\}
        \uncover<4->{&= 9 + (-6) = +3^\circ}
    \end{align*}
\end{frame}

\section{课堂练习}
\begin{frame}{课堂练习}
    \begin{enumerate}
        \item 计算:$(-4\frac{1}{2}) + 3.7 - (-2.3) + 5$
        \pause
        \item 应用题:小明先向西走5km,再向东走8km,最后向西走3km,求最终位置
    \end{enumerate}

    \vspace{1cm}
    \begin{tcolorbox}[colback=red!5!white,colframe=red!75!black]
        提示:使用"正负号"表示方向,东方为正
    \end{tcolorbox}
\end{frame}

\section{总结}
\begin{frame}{总结与作业}
    \begin{columns}
        \column{0.6\textwidth}
        \begin{block}{知识树}
            \begin{itemize}
                \item 符号处理 → 统一加法
                \item 运算顺序 → 从左到右
                \item 实际应用 → 温度/位移
            \end{itemize}
        \end{block}

        \column{0.4\textwidth}
    \end{columns}

    \begin{alertblock}{作业}
        教材P35 习题1.6:第2,4,6题\\
        选做:设计商店打折问题
    \end{alertblock}
\end{frame}

\end{document}