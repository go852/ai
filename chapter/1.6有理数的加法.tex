\documentclass[aspectratio=169]{ctexbeamer} %[t]:顶端对齐
\usetheme{Madrid} %Madrid,蓝色调为主。
\usecolortheme{beaver} %beaver
\usefonttheme{professionalfonts}

\usepackage{../universe}
\uBigPaper

\date{\today}
\begin{document}

\begin{frame}[t]{1.6 有理数的加法法则}
\begin{spacing}{1.1} %设置行距
\normalsize
\begin{enumerate}[label={\arabic*.}]
\item 同号两数相加, 取与加数相同的正负号, 并把绝对值相加; \\
$\text{当}a,b  > 0\text{时,}$ \\
$(+a) + (+b) = +(a+b) = a + b$ \\
$(-a) + (-b) = -(a+b)$ \\
\item 绝对值不相等的异号两数相加, 取绝对值较大的加数的正负号, 并用较大的绝对值减去较小的绝对值; \\
$\text{当}a > b > 0\text{时,}$ \\
$(-a) + (+b) = -(a-b)$ \\
$(+a) + (-b) = +(a-b) = a - b$ \\
\item 互为相反数的两个数相加得0; \\
 $a + (-a) = 0$ \\
\item 一个数与0 相加, 仍得这个数. \\
$a + 0 = 0 $\\
\end{enumerate}
\end{spacing}
\end{frame}

\begin{frame}[t]{有理数加法的运算律}
\begin{spacing}{1.5} %设置行距
\Large
\begin{enumerate}[label={\arabic*.}]
\item 加法交换律:两个数相加,交换加数的位置,和不变. \\
$a + b = b + a$
\item 加法结合律: 三个数相加, 先把前两个数相加,或者先把后两个数相加,和不变. \\
$(a + b) + c = a + (b + c)$
\end{enumerate}
\end{spacing}
\end{frame}

\begin{frame}[t]{有理数去括号规则}
\begin{spacing}{1.5} %设置行距
\Large
括号前的符号与数字前的符号存在下列关系,则:\\
\begin{enumerate}[label={\arabic*.}]
\item 同号去括号取正号 \\
$+(+a) = +a = a$ \\
$-(-a) = +a = a$ \\
\item 异号去括号取负号\\
$-(+a) = -a$ \\
$+(-a) = -a$ \\
\end{enumerate}
\end{spacing}
\end{frame}

\end{document}