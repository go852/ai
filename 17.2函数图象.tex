\documentclass[aspectratio=169]{ctexbeamer} %[t]:顶端对齐
\usetheme{Madrid} %Madrid,蓝色调为主。
\usecolortheme{beaver} %beaver
\usefonttheme{professionalfonts}

\usepackage{universe}
\uBigPaper

\date{\today}
\begin{document}

%17.2 函数图象
% 定义一个命令,包含 tikz 图形
\newcommand{\fig}[1][1]{
\begin{figure}
\centering
\begin{tikzpicture}[scale=#1]
  \fontsize{14pt}{16pt}\selectfont % 设置normalsize
  % 绘制坐标轴
  \draw[->, >=stealth] (-3.5,0) -- (3.5,0) node[right] {$x$};
  \draw[->, >=stealth] (0,-3.5) -- (0,3.5) node[above] {$y$};
  
  % 添加x轴刻度
  \foreach \x in {-3,-2,-1,1,2,3}
    \draw (\x,0.1) -- (\x,0) node[below] {$\x$};
  
  % 添加y轴刻度
  \foreach \y in {-3,-2,-1,1,2,3}
    \draw (0.1,\y) -- (0,\y) node[left] {$\y$};

  % 标注原点O
  \node at (0,0) [below right] {$O$};
  
  % 标注象限(使用大写罗马数字)
  \node at (2,2.5) {\textrm{I}};
  \node at (-2,2.5) {\textrm{II}};
  \node at (-2,-2) {\textrm{III}};
  \node at (2,-2) {\textrm{IV}};
  
  % 绘制点P(3,2)及其标签
  \filldraw (3,2) circle (1pt) node[above right] {$P$};
  
  % 绘制虚线投影并标注M、N点
  \draw[dashed] (3,2) -- (3,0);
  \filldraw (3,0) circle (1pt) node[above right] {$M$};
  \draw[dashed] (3,2) -- (0,2);
  \filldraw (0,2) circle (1pt) node[above right] {$N$};
\end{tikzpicture}
\caption{17.2.2}
\end{figure}
}


\begin{frame}{17.2 函数图象(平面直角坐标系)}
\begin{columns}
\column{0.5\textwidth}
\textbf{在数学中,我们可以用一对有序实数来确定平面上点的位置。}\\
\vspace{12pt}
\textbf{\textcolor{orange}{为此,在平面上画两条原点重合、互相垂直且具有相同单位长度的数轴,这就建立了平面直角坐标系(rectangle coordinate system)。}} \\
\vspace{12pt}
\textbf{通常把其中水平的数轴叫做$x$轴或横轴,取向右为正方向;铅直的数轴叫做$y$轴或纵轴,取向上为正方向;两条数轴的交点$O$叫做坐标原点。} \\
\vspace{12pt}
\textbf{为了纪念法国数学家笛卡儿,通常称为笛卡儿直角坐标系。}
\column{0.4\textwidth}
\fig[1.8]

\end{columns}
\end{frame}

\begin{frame}{平面直角坐标系}
\begin{columns}
\column{0.6\textwidth}
%\setlength{\baselineskip}{32pt}  % 设置行间距
\textbf{在平面直角坐标系中,任意一点都可以用一对有序实数来表示。例如,图17.2.2中的点$P$,从点$P$分别向$x$轴和$y$轴作垂线,垂足分别为点$M$和点$N$。}\\
\vspace{12pt}
\textbf{这时,点M在$x$轴上对应的数为3,\textcolor{orange}{称为点P的横坐标(abscissa)。}点$N$在$y$轴上对应的数为2,\textcolor{orange}{称为点$P$的纵坐标(ordinate)。}}\\
\vspace{12pt}
\textbf{依次写出点$P$的横坐标和纵坐标,得到一对有序实数(3, 2),称为点$P$的坐标。这时点$P$可记作$P(3, 2)$。}\\
\vspace{12pt}
\textbf{在平面直角坐标系中,两条坐标轴把平面分成如图17.2.2所示的I、II、III、IV四个区域,\textcolor{orange}{分别称为第一、二、三、四象限。坐标轴上的点不属于任何一个象限。}}

\column{0.4\textwidth}
\fig[1.4]
\end{columns}
\end{frame}

\end{document}
