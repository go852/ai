% 有理数的引入
\begin{frame}{1.1 有理数的引入}
\begin{definition}
\textbf{\textcolor{orange}{正整数、0 和负整数统称为整数(integer), 正分数和负分数统称为分数(fraction).\\
整数和分数统称为有理数(rational number).}}
\end{definition}
\vspace{12pt}
\begin{columns}
\column{0.5\textwidth}
\[
\mbox{有理数}\begin{cases}
\mbox{整数} \begin{cases}
    \mbox{正整数} \\
    0 \\
    \mbox{负整数}
    \end{cases} \\
\mbox{分数}  \begin{cases}
    \mbox{正分数} \\
    \mbox{负分数}
    \end{cases}
\end{cases}
\]

\column{0.5\textwidth}
\[
\mbox{小数}\begin{cases}
\mbox{有限小数} \\
\mbox{无限小数} \begin{cases} 
\mbox{无限循环小数} \\
\mbox{无限不循环小数}
\end{cases}
\end{cases}
\]

\end{columns}
\vspace{12pt}
\alert{\textbf{0既不是正数,也不是负数,是正数与负数的分界点。}} \\
\alert{\textbf{有限小数和无限循环小数是分数,无限不循环小数不是分数。}}
\end{frame}

%有理数集的表示方法
\begin{frame}{数集与有理数集}
数集的表示方法【数学A版必修第一册1.1集合的概念】:\\
集合A是小于10的自然数组成的集合,表示方法如下:\\
\begin{enumerate}[label={\arabic*.}]
\item 列举法:$A = \{0, 1, 2, 3, 4, 5, 6, 7, 8, 9\}$
\item 描述法:$A = \{x \in \mathbb{Z} | 0 \leq x < 10\}$
\end{enumerate}
\textbf{\textcolor{orange}{有理数集的表示方法:$Q = \{ x \in \mathbb{R} | x = \dfrac{q}{p}, p,q \in \mathbb{Z}, p \neq 0\} $ }}\\
数学中常见数集及其记法:\\
\begin{enumerate}[label={\arabic*.}]
\item 全体非负整数组成的集合称为非负整数集(或自然数集),记作$\mathbb{N}$.\\
\item 全体正整数组成的集合称为正整数集,记作$\mathbb{N}^*$或$\mathbb{N}_+$.\\
\item 全体整数组成的集合称为整数集,记作$\mathbb{Z}$.\\
\item 全体有理数组成的集合称为有理数集,记作$\mathbb{Q}$.\\
\item 全体实数组成的集合称为实数集,记作$\mathbb{R}$.\\
\end{enumerate}

\end{frame}

%思考有理数集的表示方法
\begin{frame}{思考有理数集的表示方法}
\textbf{为什么可以下面的方法表示有理数集? }\\
\vspace{12pt}
$Q = \{ x \in \mathbb{R} | x = \dfrac{q}{p}, p,q \in \mathbb{Z}, p \neq 0\} $
\vspace{5cm}

\end{frame}