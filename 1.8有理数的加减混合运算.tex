\begin{frame}[t]{1.8 有理数的加减混合运算}
\begin{spacing}{1.5}
\hspace*{2em} 算式(- 8) - (- 10) + (- 6) - (+ 4) 是有理数的加减混合运算, 可以按照运算顺序, 从左到右逐步计算; 也可以应用有理数的减法法则, 把它改写成( - 8) + (+ 10) + (- 6) + (- 4), 统一为只有加法运算的和式.\\
\hspace{2em} 在一个和式里, 通常把各个加数的括号和它们前面的加号省略不写. 如上式可写成省略加号的和的形式:\\
\[- 8 + 10 - 6 - 4.\] \\
\hspace*{2em} 这个式子仍可看作和式, 读作“负8、正10、负6、负4 的和”. 从运算意义看, 上式也可读作“负8 加10 减6减4”.
\end{spacing}
\end{frame}


\begin{frame}[t]{加法运算律在加减混合运算中的应用}
\begin{spacing}{1.5}
\hspace{2em} 因为有理数的加减法可以统一成加法, 所以在进行有理数的加减混合运算时, 可以适当应用加法运算律, 简化计算.\\
例2 计算: \\
(1) - 24 + 3.2 - 16 - 3.5 + 0.3; \\
(2) $0 - 21\dfrac{2}{3} + \left(+3\dfrac{1}{4}\right) - \left(-\dfrac{2}{3} \right) - \left(+\dfrac{1}{4} \right)$ . 
\end{spacing}
\end{frame}

\begin{frame}[t]{加法运算律在加减混合运算中的应用}
\begin{spacing}{1.4}
解:\\
(1) - 24 + 3.2 - 16 - 3.5 + 0.3 \\
= -24 - 16 -3.5 + 3.2 + 0.3 \\
= -40 - 3.5 + 3.5 \\
= -40. \\
(2) $0 - 21\dfrac{2}{3} + \left(3\dfrac{1}{4}\right) - \left(-\dfrac{2}{3} \right) - \left(+\dfrac{1}{4} \right)$  \\
$= - 21\dfrac{2}{3} + \left(3\dfrac{1}{4}\right) + \left(\dfrac{2}{3} \right) - \left(\dfrac{1}{4} \right)$ \\
$= - 21\dfrac{2}{3} + \left(\dfrac{2}{3} \right) + \left(3\dfrac{1}{4}\right) - \left(\dfrac{1}{4} \right)$ \\
$= -21 + 3 $\\
$= -18.$
\end{spacing}
\end{frame}