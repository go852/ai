\documentclass[aspectratio=169]{ctexbeamer} %[t]:顶端对齐
\usetheme{Madrid} %Madrid,蓝色调为主。
\usecolortheme{beaver} %beaver
\usefonttheme{professionalfonts}

\usepackage{universe}
\uBigPaper

\date{\today}
\begin{document}

% 1.4 绝对值
\begin{frame}{1.4绝对值}
\textbf{\textcolor{orange}{定义:我们把在数轴上表示数$a$的点与原点的距离叫做数$a$的绝对值, 记作|$a$|.}}
\begin{columns}
\column{0.45\textwidth}
\begin{enumerate}[label={\arabic*.}]
\item 一个正数的绝对值是它本身; 
\item 0 的绝对值是0;
\item 一个负数的绝对值是它的相反数.
\end{enumerate}

\begin{itemize}
    \item \textbf{数学表达式}:  $\left|  x  \right|$
    \item  \textbf{函数定义}: \[ f(x) = \begin{cases}
    x, x>0 \\
    0, x=0 \\
    -x, x<0
    \end{cases}
    \]
    \item \textbf{定义域}: $x \in \mathbb{R}$
    \item \textbf{值域}: $y \in \mathbb{R}, y \geq 0$
    \item \textbf{对称性}: 关于y轴对称
\end{itemize}

%绝对值函数图象
\column{0.5\textwidth}

\begin{figure}
\centering
\scalebox{1.8}{\subfile{fig/绝对值}}
\end{figure}

\end{columns}
\end{frame}

% 绝对值与相反数的比较
\begin{frame}{绝对值与相反数的比较}
\begin{columns}
%绝对值函数图象
\column{0.5\textwidth}
\textbf{绝对值的函数图象} \\
\begin{figure}
\centering
\scalebox{1.8}{\subfile{fig/绝对值}}
\end{figure}

%相反数函数图象
\column{0.5\textwidth}
\textbf{相反数的函数图象} \\
\begin{figure}
\centering
\scalebox{1.8}{\subfile{fig/相反数}}
\end{figure}

\end{columns}
\end{frame}

\end{document}
